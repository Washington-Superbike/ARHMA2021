
\begin{DoxyRefList}
\item[File \mbox{\hyperlink{_c_a_n_8h}{CAN.h}} ]\label{todo__todo000001}%
\Hypertarget{todo__todo000001}%
 Nothing? ~\newline
 ~\newline
 More Nothing? ~\newline
 ~\newline
 Nothing supreme? ~\newline
 ~\newline
 Nothing maxima.  
\item[File \mbox{\hyperlink{_c_a_n_8ino}{CAN.ino}} ]\label{todo__todo000002}%
\Hypertarget{todo__todo000002}%
 Change the \char`\"{}if CAN\+\_\+\+NODES != 0\char`\"{} to be an ifndef statement in the start. That means the check\+CAN and request\+Cell\+Voltages lines will not execute unless CAN\+\_\+\+NODES is a non-\/zero number in \mbox{\hyperlink{_main_8h}{Main.\+h}} before compiling. ~\newline
 ~\newline
 Potentially, check if the data is actually being updated by HIL. There is no real instantiation of associating the CAN variables with the messages being read in in the \mbox{\hyperlink{_c_a_n_8h_a785e095da30ce4993a186804102bf2ea}{can\+Task()}}, so maybe call on those methods to read in those values, in case it doesnt work. ~\newline
 ~\newline
 Goal 3. ~\newline
 ~\newline
 Final Goal.  
\item[File \mbox{\hyperlink{_data_logging_8h}{Data\+Logging.h}} ]\label{todo__todo000003}%
\Hypertarget{todo__todo000003}%
 Just add the LV current or the added up current to the current logging situation. Could be useful in future if we want to be tryhards and use a neural network to process currents or just to log our current and see what went wrong in the race. (Latter is way more likely) ~\newline
 ~\newline
 Goal 2. ~\newline
 ~\newline
 Goal 3. ~\newline
 ~\newline
 Final Goal.  
\item[File \mbox{\hyperlink{_data_logging_8ino}{Data\+Logging.ino}} ]\label{todo__todo000004}%
\Hypertarget{todo__todo000004}%
 Use the void pointer stuff in the add\+Record method. Fix it up. Not required, but preferred. ~\newline
 ~\newline
 Goal 2. ~\newline
 ~\newline
 Goal 3. ~\newline
 ~\newline
 Final Goal.  
\item[File \mbox{\hyperlink{_display_8h}{Display.h}} ]\label{todo__todo000005}%
\Hypertarget{todo__todo000005}%
 Make better structs for all the datatypes passed through and integrate it properly (mainly just thermistor vs non-\/thermistor to be honest) ~\newline
 ~\newline
 Make a better struct for time and integrate it in a way so that the update code isn\textquotesingle{}t nasty. ~\newline
 ~\newline
 Once the above stuff is implemented, please remove all extra methods, variables, etc. that are declared. ~\newline
 ~\newline
 Final Goal.  
\item[File \mbox{\hyperlink{_display_8ino}{Display.ino}} ]\label{todo__todo000006}%
\Hypertarget{todo__todo000006}%
 PLEASE remove the whole screen parameter situation. Change the \char`\"{}screen\char`\"{} variable in the \mbox{\hyperlink{_main_8h}{Main.\+h}} to be a \#define macro and then just check the value from here by directly including \mbox{\hyperlink{_main_8h}{Main.\+h}} and finding that value. Ez. ~\newline
 ~\newline
 Remove all unused vars and tighten up methods. ~\newline
 ~\newline
 There are WAY more macros (\#define statements) that need to be written for defining some frequently used values in display. Can\textquotesingle{}t think of them off the top of my head, but make and use those. ~\newline
 ~\newline
 Change the thermistor code to use \char`\"{}\+NUM\+\_\+\+THERMI\char`\"{} from \mbox{\hyperlink{_main_8h}{Main.\+h}} ~\newline
 ~\newline
 In the speedometer screen, add code to show angles. If the angle is near the threshold (\texorpdfstring{$>$}{>} $\sim$30 degrees), show it in red. If the angle is near safe (\texorpdfstring{$<$}{<} $\sim$30 degrees), show it in green. Add an indicator that prints out pre\+Charge statuses This variable is called hv\+\_\+state in \mbox{\hyperlink{_precharge_8h}{pre\+Charge.\+h}} basically print out this variable if possbie. This is similar to how cars print out engine flags, etc.  
\item[File \mbox{\hyperlink{_main_8h}{Main.h}} ]\label{todo__todo000007}%
\Hypertarget{todo__todo000007}%
 Based on refinements made for pre\+Charge/controls.\+ino, remove spare and redundant variables. And just generally ALL the spare variables. ~\newline
 ~\newline
 Based on the changes implemented for SoC, I would add another variable for the low-\/voltage current sneors or any other current sensors you add ~\newline
 ~\newline
 CHANGE THE NUM\+\_\+\+THERMI based on the number of thermistors that Powertain settles on. ~\newline
 ~\newline
 Final Goal.  
\item[File \mbox{\hyperlink{_main_8ino}{Main.ino}} ]\label{todo__todo000008}%
\Hypertarget{todo__todo000008}%
 Refine further. ~\newline
 ~\newline
 Look for redundant variables, pins, etc and remove. ~\newline
 ~\newline
 Change the name of pre\+Charge. It\textquotesingle{}s not just pre\+Charge now, I personally like the name \char`\"{}controls.\+ino\char`\"{} and \char`\"{}controls.\+h\char`\"{}. Make sure you change the \#includes as well if you do that. Please Please change \char`\"{}pre\+Charge\char`\"{} or whatever it\textquotesingle{}s called to not use this many global/struct variables. It doesn\textquotesingle{}t need everything globalized, it can pass them as parameters and eventually output to global variables $\ast$pre\+Charge.angle\+\_\+X and $\ast$pre\+Charge.angle\+\_\+Y. ~\newline
 ~\newline
 Github tutorial\+: ~\newline
 ~\newline
 Syed Yasir  
\item[File \mbox{\hyperlink{_precharge_8h}{Precharge.h}} ]\label{todo__todo000009}%
\Hypertarget{todo__todo000009}%
 Change the pre\+Charge\+Struct data to use less floats, etc. ~\newline
 ~\newline
 NUMBER\+\_\+\+OF\+\_\+\+LTCS needs to be changed for the REAL number of LTCs. MOTORCONTROLLER\+\_\+\+TEMP\+\_\+\+MAX might need to be changed depending. MOTOR\+\_\+\+TEMP\+\_\+\+MAX might need to be changed depending. ~\newline
 ~\newline
 Goal 3. ~\newline
 ~\newline
 Final Goal.  
\item[File \mbox{\hyperlink{_pre_charge_8ino}{Pre\+Charge.ino}} ]\label{todo__todo000010}%
\Hypertarget{todo__todo000010}%
 Goal 1. ~\newline
 ~\newline
 Goal 2. ~\newline
 ~\newline
 Goal 3. ~\newline
 ~\newline
 Final Goal. 
\end{DoxyRefList}